\documentclass{article}
\usepackage[utf8]{inputenc}

\title{CSCI 4511W Writing 4}
\date{\today}
\author{Sid Lin (linx1052) and Noah Park (park1623)}

\usepackage{natbib}
\usepackage{graphicx}

\begin{document}

\maketitle

Our project will be an analysis on common AI search algorithms, as discussed in the first third of the class. We will be using at least three methods of search: BFS, a variation of DFS, and A*. In class, we learned that these algorithms can be used to solve some very common AI problem scenarios. The particular focus of our project will be the slide puzzle problem. Both of us are currently taking this course while also enrolled in a game programming course, CSCI 4611; the slide puzzle scenario can be used to research the different methods of search while having the possibility of being published into an actual mobile game in the future. We found an interesting paper on this sliding puzzle titled, "Genetic Algorithm to Solve Sliding Tile 8-Puzzle Problem". The authors of the paper succeeded in creating a heuristic genetic algorithm that found a global solution in the search space
\citep{article}.
A second source of inspiration came from Greg Surma's "Sliding Puzzle - Solving Search Problem with Iterative Deepening A*". Since we want to test the A* algorithm, this article proved relevant to our interests. Surma walks through the process of creating the algorithm and his findings on the algorithm when implemented on the problem itself
\citep{surma_2019}.
\newline

The first step to completing this project is creating the search space and implementing the algorithms. The data structures used to model the slide puzzle can be done in a similar way to the AIMA code in any object-oriented language. Implementation of the search algorithms is also a matter of translating pseudocode into our respective language. After the implementation step, we could start testing and recording various benchmarks, such as number of moves, actual time, or size of the search space.
\newline

We have a few languages in mind when implementing the slide puzzle problem. A common game engine, Unity, uses C\# code for scripting. If we use C\#, we would be able to construct a game using the Unity engine on top of the project code. Swift is also another option because it be ported to an iOS app, but it is restricted to OSX users only. XCode, Unity, and Visual studio are all free IDEs, and this will be the only software we need.
\newline

To evaluate our solutions, we can use benchmarks for each of the algorithms along with different presets of the slide puzzle. For example, we can run the three algorithms on the same random shuffle of the board. In terms of statistics, appropriate ones would be total time, number of moves, or storage used. Expanding the board to a 4x4 or 5x5 could possibly highlight the differences in our selected algorithms. Finally, multiple heuristics can be tested out with the A* search.
\newline

We are currently working on the programming portion of the project and nearing completion. The programming will be fully completed by the end of this week. We plan on beginning the report immediately following the completion of the programming. The first week will involve work on the first 3-5 pages involving the description of the problem and why it is interesting as well as the review of the background of our problem. The following week, or potentially the same week, will involve work on the rest of the paper.
\newline

In terms of participation on the writing, we both worked equally on the first paragraph. Outside the introduction, Sydney wrote paragraphs two, three, and four while Noah found our relevant sources and wrote the fifth and participation paragraphs.

\bibliographystyle{plain}
\bibliography{references}
\end{document}
